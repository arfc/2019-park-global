%        File: arfc-beamer.tex
%     Created: Sun May 5 10:00 PM 2013 C


%\documentclass[11pt,handout]{beamer}
\documentclass[9pt]{beamer}
\usetheme[white]{Illinois}
%\title[short title]{long title}
\title[Safety analysis]{Safety Analysis of the Molten Salt Fast Reactor Fuel
Composition with Moltres}
%\subtitle[short subtitle]{long subtitle}
%\subtitle[Brief Summary]{A Brief Summary}
%\author[short name]{long name}
\author[Authors]{Sun Myung Park$^1$, Andrei Rykhlevskii$^1$, and Kathryn D.
Huff$^1$}
%\date[short date]{long date}
\date[09.24.2019]{September 24, 2019}
%\institution[short name]{long name}
\institute{
$^1$Dept. of Nuclear, Plasma and Radiological Engineering, University of
Illinois at Urbana-Champaign \\
}

%\usepackage{bbding}
\usepackage{amsfonts}
\usepackage{amsmath}
\usepackage{xspace}
\usepackage{xcolor}
\usepackage{graphicx}
\usepackage{subfigure}
\usepackage{booktabs} % nice rules for tables
\usepackage{microtype} % if using PDF
\usepackage{bigints}
\usepackage{minted}
\usepackage[absolute,overlay]{textpos}
\usepackage{tikz}
\usetikzlibrary{positioning, arrows, decorations, shapes}
\usetikzlibrary{shapes.geometric,arrows}
\definecolor{illiniblue}{HTML}{B1C6E2}
\tikzstyle{bblock} = [rectangle, draw, fill=illiniblue, 
text width=10em, text centered, rounded corners, minimum height=4em]
\tikzstyle{sbblock} = [rectangle, draw, fill=illiniblue, 
text width=7em, text centered, rounded corners, minimum height=4em]
\tikzstyle{arrow} = [thick,->,>=stealth]

\newcommand{\units}[1] {\:\text{#1}}%
\newcommand{\SN}{S$_N$}%{S$_\text{N}$}%{$S_N$}%
\DeclareMathOperator{\erf}{erf}
%I need some complimentary error funcitons... 
\DeclareMathOperator{\erfc}{erfc}
%Those icons in the references are terrible looking
\setbeamertemplate{bibliography item}[text]
%Figure numbering
\setbeamertemplate{caption}[numbered]
\setbeamerfont{caption}{size=\small}

%%%% Acronym support

\usepackage[acronym,toc]{glossaries}
\include{acros}

\makeglossaries

%try to get rid of header on title page\dots
\makeatletter
    \newenvironment{withoutheadline}{
        \setbeamertemplate{headline}[default]
        \def\beamer@entrycode{\vspace*{-\headheight}}
    }{}
\makeatother

\makeatother
\setbeamertemplate{footline}
{
  \leavevmode%
  \hbox{%
    \rightline{\insertframenumber{} / \inserttotalframenumber\hspace*{1ex}}
  }%
  \vskip0pt%
}
\makeatletter
\begin{document}
%%%%%%%%%%%%%%%%%%%%%%%%%%%%%%%%%%%%%%%%%%%%%%%%%%%%%%%%%%%%%
%% From uw-beamer Here's a handy bit of code to place at 
%% the beginning of your presentation (after \begin{document}):
\newcommand*{\alphabet}{ABCDEFGHIJKLMNOPQRSTUVWXYZabcdefghijklmnopqrstuvwxyz}
\newlength{\highlightheight}
\newlength{\highlightdepth}
\newlength{\highlightmargin}
\setlength{\highlightmargin}{2pt}
\settoheight{\highlightheight}{\alphabet}
\settodepth{\highlightdepth}{\alphabet}
\addtolength{\highlightheight}{\highlightmargin}
\addtolength{\highlightdepth}{\highlightmargin}
\addtolength{\highlightheight}{\highlightdepth}
\newcommand*{\Highlight}{\rlap{\textcolor{HighlightBackground}{\rule[-\highlightdepth]{\linewidth}{\highlightheight}}}}
%%%%%%%%%%%%%%%%%%%%%%%%%%%%%%%%%%%%%%%%%%%%%%%%%%%%%%%%%%%%%
%%--------------------------------%%
\begin{withoutheadline}
\frame{
  \titlepage
}
\end{withoutheadline}

%%--------------------------------%%
\AtBeginSection[]{
\begin{frame}
  \frametitle{Outline}
  \tableofcontents[currentsection]
\end{frame}
}

\section{Background and Motivation}
\subsection{Molten Salt Reactors}
\begin{frame}
	\frametitle{Molten Salt Reactors}
		\begin{itemize}
			\item A class of advanced nuclear reactor concepts that contain
			nuclear fuel dissolved and circulating in a molten salt coolant
			loop
			\item May also include designs with solid fuel and molten salt
			coolant
			\item Can potentially run for extended periods with minimal shutdown
			time due to online fuel reprocessing capabilities
		\end{itemize}
		\begin{figure}
			\centering
			\includegraphics[width=.5\textwidth]{./images/msr}
			\caption{Schematic diagram of a general \gls{MSR} concept.}
			\label{fig:msr}
		\end{figure}
\end{frame}

\begin{frame}
	\frametitle{Molten Salt Reactors}
		\textbf{Characteristics and challenges}
		\begin{itemize}
			\item Strong coupling between neutronics and thermal-hydraulics
			\begin{itemize}
				\item Strong density feedback in the fuel salt
				\item Stronger prompt response expected compared to
				existing LWRs
			\end{itemize}
			\item Movement of \glspl{DNP} in the molten
			salt loop
			\begin{itemize}
				\item Conventional safety analysis codes do not account for the
				delayed neutron precursor movement
			\end{itemize}
			\item Constantly evolving fuel composition across the lifetime of an
			\gls{MSR}
			\begin{itemize}
				\item Reactor safety parameters and transient response may
				change over time from start-up to equilibrium compositions
			\end{itemize}
		\end{itemize}
\end{frame}

\subsection{Moltres}
\begin{frame}
	\frametitle{Moltres}
		\textbf{What is Moltres?}
		\begin{itemize}
			\item Moltres \cite{lindsay_introduction_2018} is an application
			built on the \gls{MOOSE} framework, for the simulation of \glspl{MSR}
			\item \gls{MOOSE} \cite{gaston_moose:_2009} is an open source finite
			element framework written in \texttt{C++} that
			relies on Libmesh and PETSc for advanced meshing and PDE solving
			capabilities
			\item Moltres can run transient, implicitly coupled
			neutronics/thermal-hydraulics simulations
			\begin{itemize}
				\item Multi-group neutron diffusion (arbitrary no. of groups)
				\item \gls{DNP} decay (with advection)
				\item Incompressible Navier-Stokes for temperature
				advection-diffusion
				\item 1D, 2D, and 3D modeling are supported
			\end{itemize}
		\end{itemize}
\end{frame}
\subsection{Objectives}
\begin{frame}
	\frametitle{Objectives}
		\textbf{Objectives}
		\begin{itemize}
			\item Verify the neutronics capabilities in Moltres against Serpent
			\item Demonstrate coupled neutronics/thermal-hydraulics simulations
			of the \gls{MSFR} concept,
			\begin{itemize}
				\item with \gls{BOL}, early-life, and \gls{EOL} fuel compositions
				\item for steady state and transient cases: \gls{ULOHS} accident
			\end{itemize}
		\end{itemize}
\end{frame}

\section{Method}
\subsection{Molten Salt Fast Reactor}
\begin{frame}
	\frametitle{Molten Salt Fast Reactor}
		\textbf{Features}
		\begin{itemize}
			\item Fast-spectrum \gls{MSR} concept
			\item Designed to run on a closed thorium fuel cycle
			\item Primary fuel salt flows upwards through the central core
			region and separates into 16 smaller external loops towards the
			heat exchangers and pumps
			\item Radially surrounded by a tank of blanket salt consisting of
			fertile isotopes such as $^{232}$Th for breeding
		\end{itemize}
		\begin{columns}
			\column[t]{4cm}
			\begin{figure}
				\includegraphics[width=\textwidth]{../paper/figures/MSFR}
				\caption{\gls{MSFR} concept \cite{serp_molten_2014}.}
			\end{figure}
			\column[t]{6cm}
			{\footnotesize
			\begin{table}[t]
				\caption{\footnotesize Specifications of the \gls{MSFR} design
				\cite{serp_molten_2014}.}
				\begin{tabular}{ l r }
				\hline
				Parameter & Value \\
				\hline
				Thermal output [MW$_{\text{th}}$] & 3000 \\
				Electric output [MW$_{\text{e}}$] & 1500 \\
				Salt volume [m$^3$] & 18 \\
				Nominal flow rate [kg s$^{-1}$] & 18500  \\
				Nominal circulation time [s] & 4.0 \\
				Inlet temperature [K] & 923 \\
				Outlet temperature [K] & 1023 \\
				Blanket volume [m$^3$] & 7.3\\
				\hline
				\end{tabular}
			\label{table:msfr}
			\end{table}
			}
		\end{columns}
\end{frame}
\subsection{Moltres}
\begin{frame}
	\frametitle{Moltres}
		\textbf{Multi-group neutron diffusion}
		\begin{align}
	\frac{1}{v_g} &\frac{\partial \phi_g}{\partial t} - \nabla \cdot D_g \nabla
	\phi_g + \Sigma^r_g \phi_g \nonumber \\ 
	&= \sum^G_{g \neq g'} \Sigma^s_{g' \rightarrow g} \phi_{g'} + \chi^p_g
	\sum^G_{g'=1} (1-\beta) \nu \Sigma^f_{g'} \phi_{g'} + \chi^d_g \sum^I_i
	\lambda_i C_i \label{eq1}
		\end{align}
		
		\textbf{Delayed neutron precursor (with advection)}
		\begin{align}
	\frac{\partial C_i}{\partial t} = \sum^G_{g'=1} \beta_i \nu \Sigma^f_{g'}
	\phi_{g'} - \lambda_i C_i - \frac{\partial}{\partial z} u C_i \label{eq2}
		\end{align}
		
		\textbf{Temperature advection-diffusion}
		\begin{align}
	\rho c_{p} \frac{\partial T}{\partial t} + \nabla \cdot \big( \rho
	c_{p} \overrightarrow{u} \cdot T - k \nabla T \big) = Q_s - Q_{hx},
	\label{eq3}
		\end{align}
\end{frame}
%\subsection{Study Approach}
%\begin{frame}
	\frametitle{Study Approach}
		\textbf{1) Neutronics verification} \\
		\begin{itemize}
			\item Obtain \gls{MSFR} \gls{BOL}, early-life, and \gls{EOL} fuel
			compositions under the U/Th breeder configuration (available from
			Rykhlevskii et al. \cite{rykhlevskii_fuel_2019})
			\item Perform neutronics calculations on the Serpent 2 Monte Carlo
			transport code and generate
		\end{itemize}
\end{frame}

\section{Results}
\subsection{Neutronics Verification}
\begin{frame}
	\frametitle{Neutronics Verification}
		\begin{columns}
			\column{5cm}
				\textbf{Serpent Input Parameters}
					\begin{itemize}
						\item 200,000 neutrons per cycle
						\item 50 inactive, 500 active cycles
						\item JEFF-3.1.2 nuclear data library
						\item Six neutron energy groups, eight \gls{DNP} groups
						\item Temperatures defined from 900 K to 1200 K at 50 K
						intervals
					\end{itemize}
			\column{5cm}
				\begin{figure}
					\centering
					\includegraphics[width=\textwidth]
					{../paper/figures/reference}
					\caption{2D axisymmetric model used in Serpent. Derived from
					the \gls{MSFR} reference model
					\cite{fiorina_modelling_2014}.}
					\label{fig:reference}
				\end{figure}
		\end{columns}
\end{frame}

\begin{frame}
	\frametitle{Neutronics Verification}
		\textbf{Fuel Composition Data}
			\begin{itemize}
				\item Depletion calculations performed by Rykhlevskii et
				al. \cite{rykhlevskii_fuel_2019}
				\item 60-year depletion calculation on SCALE/TRITON
				using a unit cell representation of the \gls{MSFR} (and
				three other fast-spectrum \glspl{MSR}
				\item U/Th breeder reprocessing scheme
				\item Compositions:
				\begin{itemize}
					\item Start-up: 77.5\% LiF - 19.9\% ThF$_4$
					- 2.6\% UF$_4$
					\item Early-life: 300 days after start-up
					\item Equilibrium: ~43 years after start-up
					($<$3\% change in TRU vector between depletion
					time-steps)
				\end{itemize}
			\end{itemize}
\end{frame}

\begin{frame}
	\frametitle{Neutronics Verification}
	\begin{columns}
		\column{5.5cm}
		\textbf{Moltres Simulation Details}
		\begin{itemize}
			\small
			\item Adaptive backward Euler time-stepper
			\item Six neutron groups, eight \gls{DNP} groups
			\item Vacuum neutron boundary condition on outer edges
			\item Uniform upward flow of 1.125 m s$^{-1}$
			\item Flow/decay of \glspl{DNP} and heat removal in the outer loop
			is simulated on a simplified 1D geometry separate from active core
			region
			\item Fixed heat removal rate to secondary loop system
		\end{itemize}
		\column{4.5cm}
		\begin{figure}
			\centering
			\includegraphics[width=.9\textwidth]{../paper/figures/mesh}
			\caption{Mesh of the 2D axisymmetric model used in Moltres. The
			grey and red regions represent the fuel and blanket salt
			respectively.}
			\label{fig:mesh}
		\end{figure}
	\end{columns}
\end{frame}

\begin{frame}
	\frametitle{Neutronics Verification}
		\textbf{Results}
			\begin{figure}
				\centering
				\includegraphics[width=.8\textwidth]{../paper/figures/nt-spec}
				\caption{Fine-group and six-group neutron flux distributions
				from Serpent and Moltres (start-up composition without \gls{DNP}
				drift at 973 K).}
				\label{fig:ntspec}
			\end{figure}
\end{frame}

\begin{frame}
	\frametitle{Neutronics Verification}
		\textbf{Temperature Reactivity Feedback}
			\begin{table}[t]
				\centering
				\caption{Temperature reactivity feedback coefficients.}
				\begin{tabular}{lccc}
					\hline
					\multirow{2}{*}{Composition} & {$\alpha_T$ [pcm K$^{-1}$]} & {$\alpha_T$ [pcm K$^{-1}$]} & Difference\\
					& Serpent & Moltres & [pcm K$^{-1}$]\\
					\hline
					Start-up & $-7.39 \pm 0.03$ & $-7.46$ & $-0.07$\\
					Early-life & $-7.25 \pm 0.03$ & $-7.33$ & $-0.08$\\
					Equilibrium & $-6.24 \pm 0.03$ & $-6.34$ & $-0.10$\\
					\hline
				\end{tabular}
				\label{table:reactivity}
			\end{table}	
\end{frame}

\subsection{Steady State Case}
\begin{frame}
	\frametitle{Steady State Case}
		\textbf{Steady state neutron flux (with \gls{DNP} drift)}
		\begin{columns}
			\column{5cm}
			\begin{figure}
				\centering
				\includegraphics[width=\textwidth]{../paper/figures/totalflux}
				\caption{\small Total radial neutron flux at reactor
				half-height, for start-up, early-life, and equilibrium fuel
				compositions.}
				\label{fig:totalflux}
			\end{figure}
			\column{5cm}
			\begin{figure}
				\centering
				\includegraphics[width=\textwidth]{../paper/figures/stflux}
				\caption{\small Neutron group fluxes at reactor half-height, for
				start-up, early-life, and equilibrium fuel compositions.}
				\label{fig:stflux}
			\end{figure}
		\end{columns}
\end{frame}

\begin{frame}
	\frametitle{Steady State Case}
		\textbf{Steady state temperature distribution}
		\begin{columns}
			\column{12cm}
			\begin{figure}
				\centering
				\begin{subfigure}
					\centering
					\includegraphics[width=.3\textwidth]{../paper/figures/sttemp}
				\end{subfigure}
				\begin{subfigure}
					\centering
					\includegraphics[width=.3\textwidth]{../paper/figures/eltemp}
				\end{subfigure}
				\begin{subfigure}
					\centering
					\includegraphics[width=.3\textwidth]{../paper/figures/eqtemp}
				\end{subfigure}\\
				\hspace*{\fill}
				(a) Start-up \hfill \hfill (b) Early-life \hfill \hfill
				(c) Equilibrium
				\hspace*{\fill}
				\caption{Temperature distribution in fuel salt region for
				start-up, early-life, and equilibrium fuel compositions.}
				\label{fig:temp}
			\end{figure}
		\end{columns}
\end{frame}


\section{Conclusion}
\subsection{Conclusion}
%\begin{frame}
	\frametitle{Conclusion}
		The neutron group fluxes and reactivity coefficients from Moltres
		(without \gls{DNP} drift) are in
		good agreement with Serpent results.
		
		\vspace{.3cm}
		The steady state neutron flux, temperature, and \gls{DNP} distributions
		observed are expected for uniform flow.
		
		\vspace{.3cm}
		Discrepancies present in peak neutron
		flux and temperature distributions relative to other work
		
		\vspace{.3cm}
		Higher core temperatures observed for start-up fuel composition than
		early-life and equilibrium compositions.
\end{frame}
\subsection{Future Work}
%\input{future_work}
%\begin{frame}
	\frametitle{Acknowledgments}
    	Sun Myung Park is supported by the Singapore Nuclear Research and Safety
    	Initiative (SNRSI) Postgraduate Scholarship.
    	
    	\vspace{.3cm}
    	Andrei Rykhlevskii and Dr Kathryn Huff are both supported by funding from
    	the DOE ARPA-E MEITNER Program (award DE-AR0000983).
    	
    	\vspace{.3cm}
    	Dr Kathryn Huff is also supported by the Blue Waters sustained-petascale
    	computing project supported by the National Science Foundation.
\end{frame}

\begin{frame}
	\large
	\centering
	\textbf{Thank you for your attention!}
\end{frame}

%%--------------------------------%%
%%--------------------------------%%
\begin{frame}[allowframebreaks]
  \frametitle{References}
  \bibliographystyle{plain}
  {\footnotesize \bibliography{./bibliography.bib} }

\end{frame}

%%--------------------------------%%


\end{document}

