\documentclass{anstrans}

\title{Coupled Neutronics/Thermal-Hydraulics Simulation of the Molten Salt Fast Reactor with Moltres}
\author{Sun Myung Park$^{1,2}$ and Kathryn D. Huff$^1$}
\institute{$^1$Dept. of Nuclear, Plasma and Radiological Engineering, University of Illinois at Urbana-Champaign \\
$^2$smpark3@illinois.edu}

\usepackage{graphicx} % allows inclusion of graphics
\usepackage{booktabs} % nice rules (thick lines) for tables
\usepackage{microtype} % improves typography for PDF
\usepackage{xspace}
\usepackage{multirow} 
\usepackage{array}
\setlength{\arrayrulewidth}{.4mm}
\renewcommand{\arraystretch}{1.2}
\usepackage[labelfont=bf]{caption}
\captionsetup[table]{name=Table}
\renewcommand{\thetable}{\arabic{table}}
\usepackage{subcaption}
\usepackage{enumitem}
\usepackage{placeins}
\usepackage{siunitx}
\newcolumntype{c}{>{\hsize=.56\hsize}X}
\newcolumntype{b}{>{\hsize=.7\hsize}X}
\newcolumntype{s}{>{\hsize=.74\hsize}X}
\newcolumntype{f}{>{\hsize=.1\hsize}X}
\newcolumntype{a}{>{\hsize=.45\hsize}X}
\usepackage{titlesec}
\titleformat*{\subsection}{\normalfont}

\begin{document}

\begin{abstract}
Molten salt reactors (MSR) differ greatly from conventional solid-fuelled reactors, particularly in the neutronics and thermal-hydraulics behaviours. They present unique computational challenges that conventional computational tools fail to address effectively. The development of novel simulation tools for MSRs is an essential step towards understanding the behaviour of MSRs under various conditions. An important requirement for new MSR simulation tools is strong coupling between neutronics and thermal-hydraulics modeling due to the strong temperature reactivity coefficients arising from Doppler effects and changes in the fuel salt density.

Moltres is an open source coupled neutronics/thermal hydraulics simulation tool for simulating MSRs \cite{lindsay_introduction_2018}. It is under active development at the University of Illinois. It is an application code built using the MOOSE finite element framework \cite{gaston_moose:_2009}. Moltres is developed as a solver for the coupled multi-group neutron diffusion, temperature and delayed neutron precursor governing equations. As a multi-group neutronics solver on a coarse mesh, Moltres imposes relatively low computational loads compared to continuous energy neutronics solvers and computational fluid dynamics thermal-hydraulic solvers employed by other researchers in the study of MSRs. This allows for steady-state, transient and accident analyses with relatively low computational requirements.

This abstract presents the outline for the code-to-code verification of Moltres. The reactor model used for this verification is the Molten Salt Fast Reactor. The MSFR is a reference design for a fast-spectrum molten salt reactor developed under the EVOL and SAMOFAR projects\cite{serp_molten_2014}. There is abundant MSFR parameter and simulation data available from various authors \cite{fiorina_modelling_2014} \cite{pettersen_coupled_2016} for the reference model to perform the code-to-code verification.

Moltres requires neutron group constant data from a neutron transport code. For this study, Serpent is used to generate the group constant data. Serpent \cite{leppanen_serpent_2015} is a continuous-energy Monte Carlo code used in numerous reactor physics applications. The inputs consist of MSFR reference model parameters, neutron energy group bounds, a nuclear data library, and the temperatures at which the group constants are to be generated. Six neutron group constant data is generated for the six-group neutron diffusion equation.

Moltres interpolates the provided group constant data generated by Serpent at various discrete temperatures. Other Moltres input parameters include flow velocity, a geometry mesh file, appropriate initial values, boundary conditions and material properties. The flow velocity is essential for advective heat transfer and the movement of delayed neutron precursors in the model.

The full paper will present Moltres results of MSFR reference model under various conditions defined by the reference model, such as the temperature of the fuel at the inlet and outlets and flow velocities. Furthermore, transient scenarios such as pump over-speed will be explored and studied and verified against existing simulation data by various authors.
\end{abstract}

\bibliographystyle{ans}
\bibliography{bibliography}

\end{document}
