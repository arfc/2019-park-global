\documentclass{anstrans}

\title{Safety Analysis of the Molten Salt Fast Reactor Fuel Composition using Moltres}
\author{Sun Myung Park$^{1,2}$ and Kathryn D. Huff$^1$}
\institute{$^1$Dept. of Nuclear, Plasma and Radiological Engineering, University of Illinois at Urbana-Champaign \\
$^2$smpark3@illinois.edu}

\usepackage{graphicx} % allows inclusion of graphics
\usepackage{booktabs} % nice rules (thick lines) for tables
\usepackage{microtype} % improves typography for PDF
\usepackage{xspace}
\usepackage{multirow} 
\usepackage{array}
\setlength{\arrayrulewidth}{.4mm}
\renewcommand{\arraystretch}{1.2}
\usepackage[labelfont=bf]{caption}
%\captionsetup[table]{name=Table}
\renewcommand{\thetable}{\arabic{table}}
\usepackage{subcaption}
\usepackage{enumitem}
\usepackage{placeins}
\usepackage{siunitx}
\newcolumntype{c}{>{\hsize=.56\hsize}X}
\newcolumntype{b}{>{\hsize=.7\hsize}X}
%\newcolumntype{s}{>{\hsize=.74\hsize}X}
\newcolumntype{f}{>{\hsize=.1\hsize}X}
\newcolumntype{a}{>{\hsize=.45\hsize}X}
\usepackage{titlesec}
\titleformat*{\subsection}{\normalfont}

\begin{document}

\begin{abstract}
%
Molten salt reactors (MSR) potentially possess the ability to run for extended periods with minimal shutdown time due to online fuel reprocessing.
Their proposed equilibrium fuel compositions differ from start-up compositions due to burnup of initial fissile material, breeding of new fissile material, and production/removal of fission products.
The impact of changing fuel composition on safety parameters (e.g. reactivity feedback coefficient) must be fully characterized to establish a licensing case for this class of reactors.

Numerous computational tools exist for conventional nuclear reactors.
However, most are not suitable for the study of MSRs.
MSRs differ greatly from conventional solid-fuelled reactors, particularly in their neutronics and thermal-hydraulics behaviours.
They present unique computational challenges that many existing computational tools fail to address effectively.
New MSR simulation tools must capture strong coupling between neutronics and thermal-hydraulics exhibited by delayed neutron precursor movement as well as strong Doppler and density feedback in the fuel salt.
This paper introduces a new simulation tool, Moltres \cite{lindsay_introduction_2018}, for investigating the impact of changing fuel composition on safety parameters.

Moltres is an open source coupled neutronics/thermal hydraulics simulation tool for simulating MSRs.
It is a software package built on the MOOSE finite element framework \cite{gaston_moose:_2009}.
It solves the coupled time-dependent multi-group neutron diffusion, temperature and delayed neutron precursor (DNP) governing equations.
The temperature and DNP equations fully account for fuel advection as the fuel salt flows upwards through the core.

The MSR model studied in this paper is the Molten Salt Fast Reactor (MSFR) \cite{merle-lucotte_launching_2011}.
The MSFR is a reference design for a fast-spectrum molten salt reactor developed under the EVOL and SAMOFAR projects \cite{serp_molten_2014}.

The full paper will present Moltres simulation results of the MSFR reference model with three fuel compositions:
start-up, early life and equilibrium fuel compositions.
The chosen start-up fuel composition is a eutectic mixture of $^{233}$U and $^{232}$Th fluorides in a lithium fluoride molten salt \cite{merle-lucotte_launching_2011}.
This is in line with the main purpose of running the MSFR as a thorium breeder.
Cross-section data for these fuel compositions are generated using Serpent \cite{leppanen_serpent_2015},
which is a continuous-energy Monte Carlo code used in numerous reactor physics applications.
Moltres will solve the coupled neutron transport equation as a steady-state eigenvalue problem to obtain the effective neutron multiplication factor, $k_{eff}$.
For each fuel composition, the temperature distribution and densities will be varied slightly from the steady-state configuration to obtain the change in $k_{eff}$.
Reactivity feedback coefficients will be calculated from these variations and assessed for each fuel compositions.
Finally, the results will be compared with existing data \cite{fiorina_analysis_2012} \cite{fiorina_investigation_2013} of safety parameters for the three fuel compositions.

\end{abstract}

\bibliographystyle{ans}
\bibliography{bibliography}

\end{document}
