\documentclass{anstrans}

\title{Safety Analysis of the Molten Salt Fast Reactor Fuel Composition using Moltres}
\author{Sun Myung Park$^{1,2}$ and Kathryn D. Huff$^1$}
\institute{$^1$Dept. of Nuclear, Plasma and Radiological Engineering, University of Illinois at Urbana-Champaign \\
$^2$smpark3@illinois.edu}

\usepackage{graphicx} % allows inclusion of graphics
\usepackage{booktabs} % nice rules (thick lines) for tables
\usepackage{microtype} % improves typography for PDF
\usepackage{xspace}
\usepackage{multirow} 
\usepackage{array}
\setlength{\arrayrulewidth}{.4mm}
\renewcommand{\arraystretch}{1.2}
\usepackage[labelfont=bf]{caption}
%\captionsetup[table]{name=Table}
\renewcommand{\thetable}{\arabic{table}}
\usepackage{subcaption}
\usepackage{enumitem}
\usepackage{placeins}
\usepackage{siunitx}
\newcolumntype{c}{>{\hsize=.56\hsize}X}
\newcolumntype{b}{>{\hsize=.7\hsize}X}
%\newcolumntype{s}{>{\hsize=.74\hsize}X}
\newcolumntype{f}{>{\hsize=.1\hsize}X}
\newcolumntype{a}{>{\hsize=.45\hsize}X}
\usepackage{titlesec}
\titleformat*{\subsection}{\normalfont}


\usepackage[acronym,toc]{glossaries}
\include{acros}
\makeglossaries

\begin{document}

\begin{abstract}
%
        \glspl{MSR} potentially possess the ability to run for extended 
        periods with minimal shutdown time due to online fuel reprocessing.  
        Their proposed equilibrium fuel compositions differ substantially from 
        start-up compositions due to burnup of initial fissile material and 
        breeding of new fissile material, but also fissile material feeds and 
        removal of fission products.  Since the changing fuel composition 
        impacts safety parameters (e.g. reactivity feedback coefficients), a 
        licensing case for this class of reactors  must fully characterize 
        those impacts. 

Numerous computational tools exist for conventional nuclear reactors, but 
        \glspl{MSR} present unique computational challenges that many fail to 
        address effectively.  \glspl{MSR} differ profoundly from conventional 
        solid-fuelled reactors, particularly in their neutronics and 
        thermal-hydraulics behaviors.  New \gls{MSR} simulation tools must 
        capture strong coupling between neutronics and thermal-hydraulics 
        exhibited by delayed neutron precursor movement as well as strong 
        Doppler and density feedback in the fuel salt.  This paper investigates 
        the impact of changing fuel composition on safety parameters in the 
        \gls{MSFR}  using a new simulation tool for \glspl{MSR}, Moltres 
        \cite{lindsay_introduction_2018}.

Moltres is an open source coupled neutronics/thermal hydraulics simulation 
        application for simulating \glspl{MSR}. Built on the \gls{MOOSE} finite 
        element framework \cite{gaston_moose:_2009}, Moltres solves the coupled 
        time-dependent multi-group neutron diffusion, temperature, and 
        \gls{DNP} governing equations.  The temperature and DNP equations fully 
        account for fuel advection as the fuel salt flows upwards through the 
        core.

The \gls{MSFR} model studied in this paper
        \cite{merle-lucotte_launching_2011}, is a reference design for a 
        fast-spectrum \gls{MSR} developed under the \gls{EVOL} and 
        \gls{SAMOFAR} projects \cite{serp_molten_2014}.
        The full paper will present results from Moltres simulation of the 
        \gls{MSFR} reference model with three fuel compositions:
        start-up, early life, and equilibrium.

In line with the purpose of the \gls{MSFR} as a thorium breeder, its chosen 
        start-up fuel composition is a eutectic mixture of $^{233}$U and 
        $^{232}$Th fluorides in a lithium fluoride molten salt 
        \cite{merle-lucotte_launching_2011}. We generate group constants for 
        each fuel composition using Serpent \cite{leppanen_serpent_2015}, a 
        continuous-energy Monte Carlo code for numerous reactor physics 
        applications. For verification, we will compare group constants and
        reactivity coefficients with existing data 
        \cite{fiorina_analysis_2012,fiorina_investigation_2013} of safety 
        parameters for the three fuel compositions.  
        Using these group constants ($\chi$, $\nu\Sigma_f$, 
        $\Sigma_{g\rightarrow g'}$, $\frac{d\rho}{dT}$, etc.), Moltres then 
        solves for the flux and temperature based on the neutron diffusion 
        equation coupled with navier stokes thermal hydraulics. Transient 
        simulations will establish the spatial distribution of flux, 
        $\phi_g(\vec{r})$ and temperature, $T(\vec{r})$ during transients.
        These distributions will give insight into MSFR transient behavior,
        which will help us identify potential safety risks.
        These risks may warrant further study and possibly, changes to the
        fuel composition or the online fuel reprocessing scheme.
\end{abstract}

\bibliographystyle{ans}
\bibliography{bibliography}

\end{document}
