\documentclass{anstrans}

\title{Coupled Neutronics/Thermal-Hydraulics Simulation of the Molten Salt Fast Reactor with Moltres}
\author{Sun Myung Park$^{1,2}$ and Kathryn D. Huff$^1$}
\institute{$^1$Dept. of Nuclear, Plasma and Radiological Engineering, University of Illinois at Urbana-Champaign \\
$^2$smpark3@illinois.edu}

\usepackage{graphicx} % allows inclusion of graphics
\usepackage{booktabs} % nice rules (thick lines) for tables
\usepackage{microtype} % improves typography for PDF
\usepackage{xspace}
\usepackage{multirow} 
\usepackage{array}
\setlength{\arrayrulewidth}{.4mm}
\renewcommand{\arraystretch}{1.2}
\usepackage[labelfont=bf]{caption}
%\captionsetup[table]{name=Table}
\renewcommand{\thetable}{\arabic{table}}
\usepackage{subcaption}
\usepackage{enumitem}
\usepackage{placeins}
\usepackage{siunitx}
\newcolumntype{c}{>{\hsize=.56\hsize}X}
\newcolumntype{b}{>{\hsize=.7\hsize}X}
%\newcolumntype{s}{>{\hsize=.74\hsize}X}
\newcolumntype{f}{>{\hsize=.1\hsize}X}
\newcolumntype{a}{>{\hsize=.45\hsize}X}
\usepackage{titlesec}
\titleformat*{\subsection}{\normalfont}

\begin{document}

\begin{abstract}
%
This abstract presents the outline for the code-to-code verification of Moltres.
Moltres \cite{lindsay_introduction_2018} is an open source coupled neutronics/thermal hydraulics simulation tool for simulating molten salt reactors (MSR).
MSRs differ greatly from conventional solid-fuelled reactors, particularly in their neutronics and thermal-hydraulics behaviours.
They present unique computational challenges that many existing computational tools fail to address effectively.
Strong coupling between neutronics and thermal-hydraulics modeling is an important requirement for new MSR simulation tools due to the strong temperature reactivity coefficients arising from Doppler effects and changes in the fuel salt density. 

The development of novel simulation tools for MSRs is an essential step towards understanding the behaviour of MSRs under various conditions, one of which is the changing fuel compositions across the lifespan of MSRs.
The fuel composition in an MSR continuously evolves over time as online refueling replenishes fissile material while removing fission product poisons. The impact of changing fuel composition on the safety parameters such as the reactivity feedback coefficient must be studied extensively.

Moltres is an application code built in the MOOSE finite element framework \cite{gaston_moose:_2009}.
Moltres is developed as a solver for the coupled multi-group neutron diffusion, temperature and delayed neutron precursor governing equations.
MOOSE is a parallel framework and it has an adaptive meshing scheme.
Therefore, reactor simulations with large complex meshes can scale effectively when run on multiple cores. 

The Molten Salt Fast Reactor (MSFR) is the MSR model used for the code-to-code verification.
The MSFR is a reference design for a fast-spectrum molten salt reactor developed under the EVOL and SAMOFAR projects\cite{serp_molten_2014}.
There are abundant MSFR parameter and simulation data available from various authors \cite{fiorina_modelling_2014} \cite{pettersen_coupled_2016} for the reference model to perform the code-to-code verification.

The full paper will present Moltres simulation results of the MSFR reference model with three fuel compositions;
They correspond to start-up, early life and equilibrium fuel compositions.
The start-up fuel composition is a mix of $^{235}$U and $^{232}$Th in LiF salt.
This is in line with the main purpose of running the MSFR as a thorium breeder.
Cross-section data for these fuel compositions will be generated using Serpent.
Serpent \cite{leppanen_serpent_2015} is a continuous-energy Monte Carlo code used in numerous reactor physics applications.
Safety parameters such as reactivity feedback coefficients will be calculated and assessed for the different compositions.
The results will be compared with simulation data from existing literature.
Furthermore, transient scenarios such as pump over-speed can be explored.

\end{abstract}

\bibliographystyle{ans}
\bibliography{bibliography}

\end{document}
